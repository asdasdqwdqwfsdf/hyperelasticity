\section{Task A}
\label{sec:task-a}

The 2nd Piola-Kirchhoff stress can be obtained from free energy using
equation \eqref{eq:PK2} in \cite{Bonet2008}:
\begin{equation} \tag{6.18}
  \label{eq:PK2}
  \utilde{S} = 2 \frac{\partial \Psi}{\partial \utilde{C}} =
  2 \frac{\partial \Psi}{\partial I_{C}} \frac{\partial I_{C}}{\partial \utilde{C}} +
  2 \frac{\partial \Psi}{\partial II_{C}} \frac{\partial II_{C}}{\partial \utilde{C}}+
  2 \frac{\partial \Psi}{\partial III_{C}} \frac{\partial III_{C}}{\partial \utilde{C}}
\end{equation}
with derivatives of invariants of Green-Lagrange deformation tensor defined as
(\(J = \det \left( \utilde{F} \right)\))
\begin{align}
  \frac{\partial I_{C}}{\partial \utilde{C}} &= \utilde{I} \tag{6.19a} \\
  \frac{\partial II_{C}}{\partial \utilde{C}} &= 2 \utilde{C} \tag{6.19b} \\
  \frac{\partial III_{C}}{\partial \utilde{C}} &= J^{2} \utilde{C}^{-1} \tag{6.22}
\end{align}
Find the missing derivatives of the free energy (given in the task) with respect
to the invariants:
\begin{align}
  \frac{\partial \Psi}{\partial I_{C}} &= 
  \frac{\mu}{2} + 2 c_{2}\left( I_{C} - 3 \right) + 3c_{3}\left( I_{C} - 3 \right)^{2}\\
  \frac{\partial \Psi}{\partial II_{C}} &= 0 \\
  \frac{\partial \Psi}{\partial III_{C}} &= \frac{\lambda \ln{J} - \mu}{2 J^{2}}
\end{align}
Lastly, the expression for the 2nd Piola-Kirchhoff stress yields
\begin{equation}
  \label{eq:PK2yeoh}
  \utilde{S} = \left[ \mu + 4c_{2}\left( I_{C} - 3 \right) +
    6c_{3}\left( I_{C} - 3 \right)^{2}\right] \utilde{I} + 
  \left( \lambda \ln{J} - \mu \right) \utilde{C}^{-1}
\end{equation}
The Lagrangian elasticity tensor can be obtained using equation 
\eqref{eq:elast-tensor-Lagrange}:
\begin{equation} \tag{6.11}
  \label{eq:elast-tensor-Lagrange}
  \uutilde{C} = \frac{\partial \utilde{S}}{\partial \utilde{E}} = 
  2 \frac{\partial \utilde{S}}{\partial \utilde{C}}
\end{equation}
Expanding invariants to have explicit dependence on \(\utilde{C}\) yields
(\textit{note the overbar open product})
\begin{equation}
  \label{eq:dSdC}
  \frac{\partial \utilde{S}}{\partial \utilde{C}} = 
  \left[ 4c_{2} + 12c_{3}\left( I_{C} - 3 \right) \right] 
      \utilde{I} \otimes \utilde{I} + 
      \frac{\lambda}{2} \utilde{C}^{-1} \otimes \utilde{C}^{-1} +
      \left( \mu - \lambda \ln{J} \right) \utilde{C}^{-1} \overline{\otimes}
      \utilde{C}^{-1}
\end{equation}
Cauchy stress is a push-forward of the 2nd Piola-Kirchhoff:
\begin{equation} \tag{5.45b}
  \label{eq:cauchy-push}
  \utilde{\sigma} = J^{-1} \utilde{F} \cdot \utilde{S} \cdot \utilde{F}^{T}
\end{equation}
Figure~\ref{fig:sigma-eps} shows component of Cauchy stress plotted against
engineering strain for the situation of uniaxial strain control
\(\utilde{F} = F_{11} \ubar{\bm{e}}_{1} \otimes \ubar{\bm{E}}_{1}\).
It is evident that the curve is nonlinear.
\begin{figure}[th]
  \pgfplotstableset{
    create on use/Y/.style={create col/copy column from table={data/task_a_sig11.dat}{0}}
  }
  \centering
  \begin{tikzpicture}
    \begin{axis}[
      width = 0.95\textwidth,
      height=\axisdefaultheight,
%      tick label style={/pgf/number format/fixed},
      try min ticks=6,
      minor tick num=1,
      grid=both,
      xmin=0, xmax=1.5,
      xlabel = {\( \varepsilon_{11}\), [-]},
      ylabel = {\( \sigma_{11} \), [MPa]},
      ]
      \addplot table[y=Y,skip first n=1] {data/task_a_eps11.dat};
    \end{axis}
  \end{tikzpicture}  
  \caption{Cauchy stress component \(\sigma_{11}\) versus strain
    \(\varepsilon_{11} = F_{11} - 1\).}
  \label{fig:sigma-eps}
\end{figure}

The Matlab implementation of this task can be found in \texttt{yeoh.m}
(see section \ref{app:matlab-code}).

%%% Local Variables:
%%% mode: latex
%%% TeX-master: "../main"
%%% End:
