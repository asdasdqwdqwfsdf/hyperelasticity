\section{Task B}
\label{sec:task-b}

For a constant strain triangle (CST) element, the shape functions can be defined as
\begin{equation}
  \label{eq:shape-func}
  N_{1} = 1 - \xi - \eta, \quad N_{2} = \xi, \quad N_{3} = \eta
\end{equation}

According to equations \eqref{eq:material-shape-grad}, the \textit{material}
gradient of shape function reads
\begin{equation} \tag{9.6ab} \label{eq:material-shape-grad}
  \ubar{\nabla}_{0}N_{a} = \frac{\partial N_{a}}{\partial \ubar{\bm{X}}} = 
  \left( \frac{\partial \ubar{\bm{X}}}{\partial \ubar{\bm{\xi}}} \right)^{-T}
  \frac{\partial N_{a}}{\partial \ubar{\bm{\xi}}} =
  \left( \frac{\partial \ubar{\bm{X}}}{\partial \ubar{\bm{\xi}}} \right)^{-T}
  \cdot \ubar{\nabla}_{\xi} N_{a}, \quad
  \frac{\partial \ubar{\bm{X}}}{\partial \ubar{\bm{\xi}}} 
  = \sum_{a=1}^{n} \ubar{\bm{X}}_{a} \otimes \ubar{\nabla}_{\xi}N_{a}
\end{equation}
%
\begin{equation}
  \ubar{\nabla}_{\xi} N_{a} = \left[
    \begin{array}{c}
      \dfrac{\partial N_{a}}{\partial \xi} \\
      \dfrac{\partial N_{a}}{\partial \eta} \\
	\end{array} \right] =
  \left[
    \begin{array}{c c c}
      \dfrac{\partial N_{1}}{\partial \xi} & \dfrac{\partial N_{2}}{\partial \xi} & \dfrac{\partial N_{3}}{\partial \xi}\\
      \dfrac{\partial N_{1}}{\partial \eta} & \dfrac{\partial N_{2}}{\partial \eta} & \dfrac{\partial N_{3}}{\partial \eta}\\
	\end{array} \right]  =
  \left[
    \begin{array}{c c c}
      -1 & 1 & 0 \\
      -1 & 0 & 1 \\
	\end{array} \right]
\end{equation}
Similarly, the \textit{spatial} shape function gradients are given as
\begin{equation} \tag{9.11ab}
  \label{eq:spatial-shape-grad}
  \ubar{\nabla} N_{a} = \frac{\partial N_{a}}{\partial \ubar{\bm{x}}} =
  \left( \frac{\partial \ubar{\bm{x}}}{\partial \ubar{\bm{\xi}}} \right)^{-T}
  \cdot \ubar{\nabla}_{\xi} N_{a}, \quad
  \frac{\partial \ubar{\bm{x}}}{\partial \ubar{\bm{\xi}}} 
  = \sum_{a=1}^{n} \ubar{\bm{x}}_{a} \otimes \ubar{\nabla}_{\xi}N_{a}
\end{equation}
Lastly, the deformation gradient can be expressed as
\begin{equation} \tag{9.5}
  \label{eq:deform-grad}
  \utilde{F} = \sum_{a=1}^{n} \ubar{\bm{x}}_{a} \otimes \ubar{\nabla}_{0} N_{a}
\end{equation}

The Matlab implementation of this task can be found in \texttt{shape\_gradients.m}
(see section \ref{app:matlab-code}).

%%% Local Variables:
%%% mode: latex
%%% TeX-master: "../main"
%%% End:
