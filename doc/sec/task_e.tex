\section{Task E}
\label{sec:task-e}

The given boundary value problem has been solved using Matlab script
\texttt{plate\_with\_hole.m} (see section \ref{app:matlab-code}).
Figure \ref{fig:bvp} shows total reaction force plotted against
displacement of the top nodes and the shape of the plate before
and after the deformation.
\begin{figure}[th]
  \centering
  % Reaction force
  \begin{subfigure}[t]{\textwidth}
    \begin{tikzpicture}
      \begin{axis}[
        width = 0.95\textwidth,
        height=\axisdefaultheight,
        minor tick num=1,
        grid=both,
        xlabel = {\(u_{x}\), [mm]},
        ylabel = {Reaction force, [N]},
        xmin=0, xmax=30
        ]
        \addplot table[skip first n=1] {data/force_displacement.dat};
      \end{axis}
    \end{tikzpicture}
    \caption{}
  \end{subfigure}

  % Geometry
  \begin{subfigure}[t]{\textwidth}
    \begin{tikzpicture}
      \begin{axis}[
        width = 0.95\textwidth,
        try min ticks=6,
        enlargelimits=false,
        axis equal image,
        axis on top,
        xlabel={\(x\), [mm]},
        ylabel={\(y\), [mm]}
        ]
        \addplot graphics[xmin=0,xmax=50,ymin=0,ymax=20] {mesh};
      \end{axis}
    \end{tikzpicture}  
    \caption{}
  \end{subfigure}
  \caption{(a) Reaction force versus displacement and (b) deformed mesh.}
  \label{fig:bvp}
\end{figure}

%%% Local Variables:
%%% mode: latex
%%% TeX-master: "../main"
%%% End:
